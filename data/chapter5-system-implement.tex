% !Mode:: "TeX:UTF-8"
\chapter{系统实现}

这一部分主要阐述异构计算资源平台的具体实现过程,具体的实现细节。这一部分将以和系统设计相反的顺序,从底到上来阐述系统的实现,分别是系统环境配置,工具链构建,平台实现和接口实现。


\section{系统环境}

平台系统环境主要包括,实现平台的操作系统和硬件信息,平台依赖的底层软件安装和不同软件版本的管理方法。

\subsection{硬件环境}

本地的主机采用的是64位的Ubuntu系统,CPU为64位的intel i5 10210U,GPU为Nvidia MX250。内存为16G内存。

\subsection{软件环境}

平台需要依赖的软件众多,软件环境非常复杂,同时需要处理不同软件之间的版本依赖,平台所依赖的所有软件及其版本如下表\ref{platform_software}所示。

\begin{table}[h!]
    \centering
    \caption{软件环境}
    \label{platform_software}
    \begin{tabular}{c|c|c}
    \hline
                                 & 软件             & 版本          \\ \hline
    驱动                          & Nvidia Driver   & 460.39      \\ \hline
    \multirow{3}{*}{GPU计算库}      & Cuda           & 11.1        \\ \cline{2-3} 
                                 & Cudnn          & 8.0.5       \\ \cline{2-3} 
                                 & OpenCL         & 2.2         \\ \hline
    \multirow{4}{*}{深度学习框架}      & Pytorch        & 1.8.1+cu111 \\ \cline{2-3} 
                                 & Tensorflow     & 2.4.1       \\ \cline{2-3} 
                                 & Keras          & 2.4.0       \\ \cline{2-3} 
                                 & Onnx           & 1.8.1       \\ \hline
    \multirow{5}{*}{Android SDK} & Emulator       & 30.5.4      \\ \cline{2-3} 
                                 & Build Tools    & 27.0.3      \\ \cline{2-3} 
                                 & Platform Tools & 31.0.1      \\ \cline{2-3} 
                                 & Cmdline Tools  & 3.0         \\ \cline{2-3} 
                                 & Ndk            & 22.1        \\ \hline
    \multirow{6}{*}{构建工具}        & GCC            & 9.3.0       \\ \cline{2-3} 
                                 & CMake          & 3.16.3      \\ \cline{2-3} 
                                 & Make           & 4.2.1       \\ \cline{2-3} 
                                 & Gradle         & 4.4.1       \\ \cline{2-3} 
                                 & Maven          & 3.6.3       \\ \cline{2-3} 
                                 & LLVM           & 10          \\ \hline
    JAVA                         & JDK            & 8           \\ \hline
    \end{tabular}
\end{table}

对于Nvidia Driver和Cuda和Cudnn,从Nvidia的官网下载下载器进行安装,在安装显卡驱动前,要在BIOS中关闭安全启动的选项。同时要关闭nvidia-drm模块,具体的安装命令如下:

\begin{lstlisting}[
    language={},
    caption={安装Cuda Toolkit},
    label={install_cuda}
]
# 关闭图形界面
sudo systemctl isolate multi-user.target
# 关闭nvidia-drm模块
sudo modprobe -r nvidia-drm
# 执行安装文件
sudo sh install.sh
# 启动图形界面
sudo systemctl start graphical.target
\end{lstlisting}

对于GCC,CMake,Make,Maven,Gradle等构建工具使用Ubuntu的包管理器进行安装,使用命令sudo apt install <package>进行安装。

对于Android SDK等软件,使用Android的sdkmanager进行安装,使用命令sdkmanager \-\-list来查看可安装的包,使用sdkmanager \-\-install <package>来安装所需的软件包。

对于深度学习框架,使用Python的包管理器pip进行安装,使用命令pip install <package>进行安装。


\subsection{软件配置}

安装软件后,平台还需要对软件进行一些配置,如设置一些环境变量来保证软件的正常使用以及多个版本共存的问题。

对于Cuda Toolkit来说,把Cuda Toolkit的文件放在系统的/usr/local/cuda/cuda-version目录下,不同的Cuda Toolkit版本放在不同的cuda-version目录下,同时设置一个链接文件current指向当前使用的版本,可以方便的进行版本的管理,同时设置CUDA\_HOME=/usr/local/cuda/current环境变量,同时需要把Cuda Toolkit的库文件目录加入LD\_LIBRARY\_PATH环境变量中,保证连接器ld能够找到Cuda Toolkit的动态连接库,把Cuda Toolkit的可执行文件目录加入PATH环境变量中。

对于JDK来说,采用和Cuda Toolkit相同的管理方法,把不同的JDK版本放在/usr/local/jdk/jdk-version目录下,设置链接文件指向当前使用的版本。同时需要指定环境变量JAVA\_HOME=/usr/local/jdk/current,同时把JDK的可执行文件目录加入PATH环境变量中。

对于Android SDK来说,需要指定环境变量ANDROID\_SDK\_ROOT=/usr/local/android-sdk。使用sdkmanager安装的ndk,build-tools,platform-tools,cmdline-tools,emulator等软件包会安装到这个目录下。


\section{工具链构建}

平台依赖的TVM和Android Rpc App以及VTA模拟器需要从源码进行编译和安装,这一部分分别阐述TVM,Android App和VTA模拟器的编译流程。

\subsection{构建TVM}

这部分给出TVM工具从源码具体的编译环境和编译步骤。


\subsubsection{编译环境}

TVM的构建环境采用上述本地电脑的环境,采用64位Ubuntu系统,intel i5 10210U CPU,Nidia MX250 GPU。依赖的Cuda版本为11.1,Cudnn版本为8.0.5,OpenCL的版本为2.2,LLVM版本为10。编译工具GCC版本为9.3.0,CMake版本为3.16.3,Make版本为4.2.0。同时按照上述环境搭建所述设置环境变量,保证在编译时能够找到Cuda,Cudnn,OpenCL,LLVM的动态连接库。


\subsubsection{编译流程}

首先从TVM的github仓库使用命令git clone --recursive <repo>下载TVM的源代码。

其次进入TVM源代码的目录,创建构建目录并设置编译配置文件,在源代码的根目录下创建build构建目录,使用命令cp cmake/config.cmake build将配置文件复制到build目录中,之后修改配置文件,使编译的动态链接库链接指定的动态链接库,具体的配置如下\ref{compile_tvm_config}:

\begin{lstlisting}[
    language={},
    caption={编译TVM配置},
    label={compile_tvm_config}
]
set(USE_CUDA ON)
set(USE_CUDNN ON)
set(USE_OPENCL ON)
set(USE_LLVM ON)
set(USE_RPC ON)
set(USE_GRAPH_EXECUTOR ON)
set(USE_PROFILER ON)
\end{lstlisting}

第三,使用如下命令\ref{compile_tvm_command}编译TVM的动态链接库,编译完成后,会在build目录下出现libtvm.so和libtvm\_runtime.so等文件。:

\begin{lstlisting}[
    language={},
    caption={编译TVM命令},
    label={compile_tvm_command}
]
cd build
cmake ..
make -j4
\end{lstlisting}

最后,把TVM的路径加入到PYTHONPATH环境变量中,使得Python的解释器能够找到TVM包。


\subsection{构建Android Rpc App}

这部分给出,Android Rpc App的编译环境,编译步骤,以及将得到的Apk文件安装到Android手机的具体细节。


\subsubsection{编译环境}

该App的编译环境采用64位Ubuntu系统,intel i5 10210U CPU,Nvidia MX250 GPU。依赖的JDK版本为8。Android SDK的Build Tools版本为27.0.3,Ndk的版本为22.1。

\subsubsection{编译流程}

首先,构建TVM的JAVA前端的包,该App依赖了TVM的JAVA前端,在TVM源代码的根目录下,使用make jvmpkg来构建这个包,这个命令会使用Maven进行构建,构建完成后,使用命令make jvminstall来把这个包安装到Maven仓库中。

其次,构建Android Rpc App,使用命令cd apps/android\_rpc来进入这个App的根目录,因为Google修改了Maven仓库,所以在build.gradle文件中把Maven的仓库修改为https://dl.google.com/dl/android/maven2/。之后创建local.properties文件,文件内容为sdk.dir=/path/to/android-sdk来指定Android SDK的路径。

第三,创建构建App的配置文件,该配置文件的具体内容如下\ref{compile_android_config}:

\begin{lstlisting}[
    language={},
    caption={编译安卓应用配置},
    label={compile_android_config}
]
APP_ABI = arm64-v8a
APP_PLATFORM = android-24
USE_OPENCL = 1
# 加入OpenCL的头文件目录
ADD_C_INCLUDES = /usr/include/CL
\end{lstlisting}

第四,使用命令gradle clean build来构建该App。

\begin{figure}[h!]
    \centering
    \includegraphics[width=180.bp]{figure/android_app.png}
    \caption{Android Rpc App界面}
    \label{android_rpc_app}
\end{figure}

最后,使用adb install <apk>来安装此App到Android设备上,安装后的应用如下图\ref{android_rpc_app}所示。


\subsection{构建VTA模拟器}
该部分介绍VTA模拟器的编译环境,编译步骤以及VTA的配置参数。

\subsubsection{编译环境}

VTA模拟器的编译环境,采用64位Ubuntu系统,intel i5 10210U CPU,Nvidia MX250 GPU。依赖的Cuda版本为11.1,Cudnn版本为8.0.5,LLVM版本为10。编译工具GCC版本为9.3.0,CMake版本为3.16.3,Make版本为4.2.0。

\subsubsection{编译流程}

首先,需要编译TVM的动态链接库,同时需要按照如下代码\ref{vta_environment}设置环境变量:

\begin{lstlisting}[
    language={bash},
    caption={VTA环境变量},
    label={vta_environment}
]
export TVM_PATH=/path/to/TVM
export VTA_HW_PATH=/path/to/TVM/3rdparty/vta-hw
\end{lstlisting}

其次,需要编译VTA的动态链接库,按照如下代码\ref{vta_compile}来进行动态链接库的编译:

\begin{lstlisting}[
    language={},
    caption={VTA编译流程},
    label={vta_compile}
]
cd <tvm-root>
mkdir build
cp cmake/config.cmake build/.
echo 'set(USE_VTA_FSIM ON)' >> build/config.cmake
cd build
cmake ..
make -j4
\end{lstlisting}

最后,通过修改PYTHONPATH环境变量,将VTA的Python前端代码加入到Python的包查找路径中,来进行使用。

\subsubsection{VTA参数}

VTA作为一个可自定义的深度学习加速器,可以通过修改一些配置参数来改变硬件的配置。本论文中用到的参数如下,其中具体的值为取对数后的值:
\begin{itemize}
    \item {TARGET:VTA部署的硬件设备,可以是sim(模拟器)或FPGA。本论文中使用sim。}
    \item {LOG\_INP\_WIDTH:输入向量的宽度。本论文中使用3。}
    \item {LOG\_WGT\_WIDTH:权重向量的宽度。本论文中使用3。}
    \item {LOG\_ACC\_WIDTH:累加器的数据宽度。本论文中使用5。}
    \item {LOG\_BATCH:VTA矩阵乘法元语中输入输出向量的维度0。本论文中使用0。}
    \item {LOG\_BLOCK:VTA矩阵乘法元语中输入输出向量的维度1。本论文中使用4。}
    \item {LOG\_UOP\_BUFF\_SIZE:微指令缓存的大小以比特为单位。本论文中使用15。}
    \item {LOG\_INP\_BUFF\_SIZE:输入向量缓存的大小以比特为单位。本论文中使用15。}
    \item {LOG\_WGT\_BUFF\_SIZE:权重向量缓存的大小以比特为单位。本论文中使用18。}
    \item {LOG\_ACC\_BUFF\_SIZE:累加器缓存的大小以比特为单位。本论文中使用17。}
\end{itemize}


\section{平台实现}

该部分具体介绍异构计算平台的各个模块以及模块的实现细节。

\subsection{编译模块}

使用TVM对多种框架的模型进行编译,同时使用具体的框架的接口来进行模型文件的加载,得到模型的对象后使用TVM的接口对模型对象进行编译得到中间表示的计算图和模型的参数。

本平台分别对支持的每种深度学习框架实现一个编译其模型的函数,本平台支持的编译函数如下:
\begin{itemize}
    \item {compile\_pytorch}
    \item {compile\_tensorflow}
    \item {compile\_keras}
    \item {compile\_onnx}
\end{itemize}

这些函数的参数都相同,如下所示:
\begin{itemize}
    \item {model:用户待编译的模型对象。}
    \item {input\_shape:模型输入向量的形状。}
    \item {kwargs:用户输入的其他参数。}
\end{itemize}

下面以compile\_pytorch函数为例,展示函数内部的编译的具体细节。

\begin{lstlisting}[
    language={Python},
    caption={compile\_pytorch函数},
    label={compile_pytorch}
]
def compile_pytorch(model: torch.nn.Module, input_shape: list, kwargs):
    input_name = 'input0'
    input_data = torch.randn(input_shape)
    
    model.eval()
    # 调用Pytorch的接口,把模型转换为Torch Script格式
    scripted_model = torch.jit.trace(model, input_data).eval()

    shape_list = [(input_name, input_shape)]

    # 调用TVM的编译API进行模型的编译
    mod, params = relay.frontend.from_pytorch(scripted_model, input_infos=shape_list, **kwargs)

    return mod, params
\end{lstlisting}


\subsection{部署模块}

该部分分别介绍部署到本地,安卓设备,树莓派和VTA的具体实现细节。


\subsubsection{部署到本地}

TVM把得到的计算图的中间表示和参数部署到本地需要指定部署的target\_host和target,target\_host用来指定在CPU端执行的代码,一般为llvm,target用来指定具体的计算平台,在本地可以使用CPU计算或者GPU计算,所以target可以为llvm或cuda。下面为具体的代码\ref{deploy_local}:

\begin{lstlisting}[
    language={Python},
    caption={部署到本地},
    label={deploy_local}
]
# 使用CPU进行计算
target = 'llvm'
target_host = 'llvm'
ctx = tvm.cpu(0)

# 得到动态链接库
with tvm.transform.PassContext(opt_level=3):
lib = relay.build(mod, target=target, target_host=target_host, params=params)
# 通过TVM执行动态链接库
m = graph_runtime.GraphModule(lib['default'](ctx))
m.set_input(input_name, input)
m.run()
tvm_output = m.get_output(0)
\end{lstlisting}


\subsubsection{部署到安卓设备}

部署到安卓与部署到本地不同,部署到安卓,需要让本地电脑能够与安卓设备进行通信,能够把得到的动态链接库上传到安卓设备,让动态链接库在安卓设备上执行。采用Rpc来使本地电脑和安卓进行通信,首先在本地,使用命令python -m tvm.exec.rpc\_tracker --host=<host> --port=<port>来创建一个Rpc的服务端。在安卓设备上,使用前文构建安装的App,输入指定的ip和端口,这样就可以使安卓设备和本地进行通信。使用TVM部署到安卓,同样需要指定target\_host和target,target\_host是在安卓端的CPU执行的代码,通常为llvm,target是安卓端的计算平台,可以是CPU,OpenCL和Vlukan。其次,在创建可以在安卓端执行的动态链接库时,因为安卓端的架构和本地不同,所以不能使用本地的编译器,需要使用TVM的ndk工具进行编译,然后将此上传到安卓端,下面是具体的代码\ref{deploy_android}:

\begin{lstlisting}[
    language={Python},
    caption={部署到安卓},
    label={deploy_android}
]
with tvm.transform.PassContext(opt_level=3):
lib = relay.build(mod, target=target, params=params)

# 使用ndk来编译为动态链接库
fcompile = ndk.create_shared
lib.export_library('models/net.so', fcompile)

# 创建一个Rpc tracker来进行通信
tracker_host = os.environ.get('TVM_TRACKER_HOST', '0.0.0.0')
tracker_port = int(os.environ.get('TVM_TRACKER_PORT', '9190'))
key = 'android'
tracker = rpc.connect_tracker(tracker_host, tracker_port)
remote = tracker.request(key, priority=0, session_timeout=60)

# 上传动态链接库到Android端
remote.upload('models/net.so')
rlib = remote.load_module('net.so')

# 指定使用OpenCL进行计算
dev = remote.cl(0)
module = runtime.GraphModule(rlib['default'](dev))
module.set_input(input_name, tvm.nd.array(x.astype(dtype)))
module.run()
out = module.get_output(0)
\end{lstlisting}

\subsubsection{部署到树莓派}

树莓派是基于Linux的单片机,处理器采用ARM架构。与部署在安卓设备相同,同样需要使用RPC与树莓派设备进行通信,在本地使用TVM编译出可执行代码后,使用RPC把可执行代码上传到树莓派设备,然后在树莓派设备上进行运行,具体的代码如下\ref{deploy_rasp}:

\begin{lstlisting}[
    language={Python},
    caption={部署到树莓派},
    label={deploy_rasp}
]
with tvm.transform.PassContext(opt_level=3):
    lib = relay.build(func, tvm.target.arm_cpu('rasp3b'), params=params)

remote = rpc.conncet(host, port)
lib_fname = lib.export_library('net.tar')
remote.upload(lib_fname)
rlib = remote.load_module('net.tar')

dev = remote.cpu(0)
module = runtime.GraphModule(rlib["default"](dev))

module.set_input("data", tvm.nd.array(x.astype("float32")))
module.run()
out = module.get_output(0)
top1 = np.argmax(out.asnumpy())
print("TVM prediction top-1: {}".format(synset[top1]))
\end{lstlisting}

\subsubsection{部署到VTA}

通过TVM编译出VTA指令集的可执行代码,得到可执行代码后,在本地的VTA模拟器执行该代码。与部署到其他设备不同的是,VTA的选项中,只支持输入和权重的向量为8位,所以在生成代码之前还需要对模型进行量化,具体的代码如下\ref{deploy_vta}:

\begin{lstlisting}[
    language={Python},
    caption={部署到VTA},
    label={deploy_vta}
]
remote = rpc.LocalSession()
with tvm.transform.PassContext(opt_level=3):
    with relay.quantize.qconfig(global_scale=8.0, skip_conv_layers=[0]):
        mod = relay.quantize.quantize(mod, params=params)
    relay_prog = graph_pack(
        mod["main"],
        env.BATCH,
        env.BLOCK_OUT,
        env.WGT_WIDTH,
        start_name=pack_dict[model][0],
        stop_name=pack_dict[model][1],
        device_annot=(env.TARGET == "sim"),
    )
with vta.build_config(opt_level=3, disabled_pass={"AlterOpLayout"}):
    graph, lib, params = relay.build(
        relay_prog, target=target, params=params, target_host=env.target_host
    )
\end{lstlisting}


\section{接口实现}

对于平台的接口实现来说,通过一个基类BaseTVMHelper来表示共通的编译部署流程,对于每种具体的设备,通过继承该基类来实现一个针对具体设备的编译部署流程,对于本平台支持的4种设备,共实现如下4个类:
\begin{itemize}
    \item {LocalTVMHelper}
    \item {AndroidTVMHelper}
    \item {RaspTVMHelper}
    \item {VtaTVMHelper}
\end{itemize}

\begin{table}
    \centering
    \caption{接口基本参数}
    \label{interface_base_params}
    \begin{tabular}{c|c|c}
        \hline
        变量                      & 类型     & 含义        \\ \hline
        device                  & string & 部署到的设备    \\ \hline
        frame                   & string & 模型使用的框架   \\ \hline
        model                   & object & 模型对象      \\ \hline
        input\_shape            & list   & 模型输入的形状   \\ \hline
        opt\_level              & int    & 优化等级      \\ \hline
        target                  & string & 设备上的计算平台  \\ \hline
        compile\_arg\_dict=\{\} & dict   & 额外的模型编译参数 \\ \hline
    \end{tabular}
\end{table}

上表\ref{interface_base_params}是接口的基本参数,其中,device是部署到的设备,目前支持local,android,rasp,和vta。frame是模型的框架,目前支持pytorch,tensorflow,keras,和onnx。model是具体框架加载后的模型对象。input\_shape是模型输入的形状。opt\_level是模型优化的等级,目前支持1,2,3。target是具体设备上的计算平台,部署到不同设备上target的值不同,如部署到local,target为cpu或cuda。部署到android设备上,target为cpu,opencl,或vulkan。

部署到本地的接口和部署到VTA的接口参数与基本参数相同,部署到安卓设备和树莓派设备上还需要一些额外的参数,如下表\ref{interface_extra_params}所示。

\begin{table}
    \centering
    \caption{接口额外参数}
    \label{interface_extra_params}
    \begin{tabular}{c|c|c}
        \hline
        变量            & 类型     & 含义                 \\ \hline
        tracker\_host & string & 创建Rpc Tracker的host \\ \hline
        tracker\_port & int    & 创建Rpc Tracker的端口   \\ \hline
        key           & string & 连接Rpc的key          \\ \hline
    \end{tabular}
\end{table}

下面通过调用本接口来把Pytorch模型部署到本地来展示接口的具体使用:

\begin{lstlisting}[
    language={Python},
    caption={本地接口使用},
    label={use_interface_local}
]
import torch
import torchvision
import numpy as np
# 引入Helper类
from PlatfromTVM import LocalTVMHelper

# 获得模型
model = getattr(torchvision.models, 'resnet18')(pretrained=True)

# 部署到local设备,使用CPU计算
deployed_model = LocalTVMHelper(device='local', frame='pytorch'
   model=model, input_shape=[1,3,224,224], opt_level=3,target='cpu')
# 通过部署的模型计算
out = deployed_model(input)
\end{lstlisting}

下图\ref{compare}是使用该平台和不使用该平台把模型部署到安卓设备上的代码对比,左侧是使用该平台,右侧是不使用该平台:

\begin{figure}[h!]
    \centering
    \includegraphics[width=270bp]{figure/compare.png}
    \caption{代码对比}
    \label{compare}
\end{figure}

可以看到,使用该平台大大简化了模型编译部署的难度,同时使用户不需要了解使用TVM编译部署的具体细节以及使用Rpc通信的具体细节。