% !Mode:: "TeX:UTF-8"
% 任务书中的信息
%% 原始资料及设计要求
\assignReq
本论文的设计技术要求是设计并实现一个基于TVM的异构计算平台。该系统的主要目的是基于TVM能够实现不同深度学习框架的模型部署到异构的硬件设备上。用户通过该系统提供的统一的接口进行模型的部署,而不需要了解其中的细节,降低模型部署的难度,提高模型部署的效率。

%% 工作内容
\assignWork
全面了解深度学习编译器的发展现状以及用户对于模型部署的需求。

根据TVM的文档和源码学习并实践模型部署的流程。

设计并实现本系统。

测试系统。

%% 参考文献
\assignRef
[1] LECUN Y, BENGIO Y, HINTON G. Deep learning[J]. nature, 2015, 521(7553):436­444.

[2] CHEN T, MOREAU T, JIANG Z, et al. Tvm: end­ to ­end optimization stack for deep learning[J]. arXiv preprint arXiv:1802.04799, 2018, 11:20.

[3] LI M, LIU Y, LIU X, et al. The deep learning compiler: A comprehensive survey[J]. IEEE Transactions on Parallel and Distributed Systems, 2020, 32(3):708­727.

[4] MOREAU T, CHEN T, JIANG Z, et al. Vta: an open hardware­software stack for deep learning[J]. arXiv preprint arXiv:1807.04188, 2018.