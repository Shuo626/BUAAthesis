% !Mode:: "TeX:UTF-8"
\chapter{需求分析}

为了使得本平台更加通用,本篇论文分别对目前深度学习开发语言,深度学习开发流程,模型的部署流程,模型在具体硬件的使用情况,模型需要的优化程度,以及类似功能的软件的设计进行了充分的调研。基于以上充分的调研后,对本平台进行了需求的分析,本平台的需求分析遵从以下的逻辑。用户需求,主要根据用户在日常使用遇到的困难来分析用户想要做什么。产品需求,基于用户的需求来分析产品需要实现怎样的功能,达到怎样的效果来满足用户的需求。产品功能需求,也是和产品的实现联系在一起的,基于产品的需求,同时结合用户的一些主流开发语言,开发流程等因素,决定产品需要实现哪些功能,如产品需要使用哪种语言进行实现,产品需要能够支持哪些框架和哪些设备,产品最终的实现形式等。


\section{用户需求}

在深度学习具体的应用场景中,训练好的神经网络模型往往需要在具体的设备上执行,如服务器,移动手机,嵌入式设备等等。就导致了用户训练模型和执行模型的设备的架构不同。所以,用户面临的问题就是需要把训练好的模型部署到不同的设备上,在部署过程中还需要针对目标设备的架构进行模型的调整和优化。 所以,用户希望能够简化模型部署的流程,同时保证模型的准确率。


\section{产品需求}

为了满足用户的需求,产品需要实现一个平台来帮助用户进行模型的部署。考虑到目前的神经网络模型多采用Python进行开发,所以本平台应该要以Python库的形式进行实现,方便用户在其项目中进行使用。其次,平台应该有应用的广泛性,即平台应该能够支持目前主流的深度学习框架,同时应该支持部署到主流的硬件设备上。最后,平台要保证易用性,保证用户简单,容易的部署模型。


\section{产品功能需求}

首先, 平台应该实现为Python的库,同时实现模型部署的API,实现的API要保证统一,简洁,方便用户调用。

其次,平台需要支持Pytorch,Tensorflow,Keras,Onnx等深度学习框架。以及服务器,移动手机,VTA等硬件设备。

最后,平台要能够对模型进行面向计算图和硬件架构的优化,保证部署模型的准确率。