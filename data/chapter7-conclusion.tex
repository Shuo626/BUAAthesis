% !Mode:: "TeX:UTF-8"
\chapter*{结论\markboth{结论}{}}


该部分将总结基于TVM的异构计算资源的设计结构和成果,以及对本平台仍存在的一些缺陷和不足进行总结。

本论文基于TVM深度学习编译器,以把不同深度学习框架的模型部署到不同的硬件设备为目标,实现了一个异构的计算平台。本平台提供了一个简单,统一的接口,并且能够支持主流的深度学习框架Pytorch,Tensorflow,Keras,Onnx,和主流的异构硬件设备,服务器,移动手机,树莓派和VTA模拟器等。

首先,本论文对目前深度学习的开发流程,部署流程进行了充分的调研,掌握了神经网络模型部署的需求以及痛点。其次,本论文对异构计算平台的研究现状进行了充分的研究,了解了目前研究中的一些优点和不足。根据用户的需求和研究现状的调查,本论文确定了研究的技术基础,并对异构计算平台进行了设计。最后,对本平台进行了实现并进行了充分的测试。达到了预期设定的目标。

本平台采用Python语言进行实现,基于TVM深度学习编译器进行模型的编译,采用RPC网络通信协议来进行不同设备间的进程通信。本平台共分为三个部分,编译模块,部署模块以及接口。通过本平台,用户能够通过一个接口来进行模型的部署,而不需要了解具体的模型编译部署,以及网络通信的细节。

就目前而言,本系统还存在着一定的缺陷。首先,本平台的设计目标是模型的基础的部署,没有提供方便的方式使得用户能够对具体硬件进行复杂的算子优化。其次,本平台还不能提供方便的方式使得用户对可支持框架和硬件进行扩展。最后,平台只对部分的常用模型进行了测试,没有能够对所有的模型进行测试。